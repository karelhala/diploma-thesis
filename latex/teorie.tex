%------------------------------------------------------------------------------------------------------------------------------------------
%  Teorie
%------------------------------------------------------------------------------------------------------------------------------------------
\chapter{TEORETICKÁ VÝCHODISKA}
\par Pro plné pochopení, výběru a případném vypracování platformz je potřeba si objasnit a vysvětlit několik témat. Jsou to především \textit{Vývojové platformy low-code}, výsledná platforma by měla splňovat tuto definici. Dále si objasníme pojemy \textit{Bussiness intelligence} (platforma bude z části pracovat s touto oblastí) a \textit{Platforma pro pokročilou vizualizaci dat} -- pro snadné používání uživatelského rozhraní. A vzhledem k tomu že výsledná platforma musí do určité části pracovat s uživatelskými právy a spravovat uživatele, objasníme si pojem \textit{Server pro řízení přístupu a identity}

\section{Vývojové platformy low-code}
http://www.pcmag.com/article/345661/building-an-app-with-no-coding-myth-or-reality
http://www.cio.com/article/2845378/development-tools/use-low-code-platforms-to-develop-the-apps-customers-want.html
http://sdtimes.com/low-code-development-seeks-accelerate-software-delivery/

\section{Bussiness Intelligence}

\subsection{Big data}

\subsection{Datamining}

\subsection{Extraction, Transaction, Loading}

\subsection{NoSql databáze}
\cite{nosql}

\section{Platforma pro pokročilou vizualizaci dat}
\subsection{Single page aplikace}
\subsection{Dynamická a interaktivní vizualizace dat}
\subsection{Webová služba RESTful}
Restufl web APIs

\section{Server pro řízení přístupu a identity}
\url{http://www.sersc.org/journals/IJMUE/vol9_no9_2014/9.pdf}

\subsection{JSON Web Token}
https://tools.ietf.org/html/rfc7519
https://scotch.io/tutorials/the-anatomy-of-a-json-web-token
