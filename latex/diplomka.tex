\documentclass[12pt,oneside,a4paper]{report}

\usepackage{ucs}
\usepackage[utf8]{inputenc}
	\PrerenderUnicode{ěščřžýáíéĚŠČŘŽÝÁÍÉďťňĎŤŇůúÚóÓ}

\usepackage[czech]{babel}
\usepackage[IL2]{fontenc}
\usepackage[automake, nogroupskip]{glossaries}
\makeglossaries
\loadglsentries{zkratky}

\usepackage{ulem}
\usepackage{parskip}
\usepackage{pdfpages}
\usepackage{graphicx}
\usepackage{float}
\usepackage[compact]{titlesec}
\usepackage{hyperref}
\usepackage{svg}
\usepackage[export]{adjustbox}
\usepackage{longtable}
\usepackage{xcolor,colortbl}

\usepackage[top=2.5cm, bottom=3.5cm, left=3.5cm, right=2.5cm]{geometry}
\usepackage{graphicx}
\graphicspath{ {fig/} }
\renewcommand{\baselinestretch}{1.5}
\newcommand{\blank}[1]{\hspace*{#1}}

\DeclareCaptionType{equationcap}[][Seznam rovnic]
\captionsetup[equationcap]{name=Rovnice}
\DeclareCaptionType{graph}[][Seznam grafů]
\captionsetup[graph]{name=Graf}

\usepackage{filecontents}
\usepackage{pgfplots, pgfplotstable}

\usepackage{afterpage}
\newcommand\blankpage{
    \null
    \thispagestyle{empty}
    \newpage
}

\titleformat{\chapter}[display]
    {\normalfont\Large\bfseries}{\chaptertitlename\ \thechapter\Makeuppercase}{20pt}{\Large}
\titlespacing*{\chapter}{0pt}{0pt}{10pt}

\titleformat*{\section}{\large\bfseries}
\titleformat*{\subsection}{\normalsize\bfseries}

\makeatletter
\renewcommand{\@makechapterhead}[1]{
{\setlength{\parindent}{0pt} \raggedright \normalfont
\bfseries\Large\thechapter\ #1
\par\nobreak\vspace{10 pt}}}
\makeatother
\setlength{\parindent}{2em}
\setlength{\parskip}{0cm}
%------------------------------------------------------------------------------------------------------------------------------------------
%  Začátek dokumentu
%------------------------------------------------------------------------------------------------------------------------------------------
\begin{document}
\afterpage{\blankpage}

%------------------------------------------------------------------------------------------------------------------------------------------
%  Titulní list
%------------------------------------------------------------------------------------------------------------------------------------------
\pagestyle{empty}
\includepdf[pages={1}]{TitulniList_color.pdf}
\includepdf[pages={1,2}]{zadani.pdf}

%------------------------------------------------------------------------------------------------------------------------------------------
%  Abstrakt
%------------------------------------------------------------------------------------------------------------------------------------------
\section*{Abstrakt}
Tato diplomová práce se zabívá vývojem platformy pro ulehčení práce s velkým množstvím dat. V rámci této práce jsou vysvětleny některé pojmy a technické věci potřebné pro bližší pochopení vývoje webové aplikace. Dále jsou zde navrhnuty postupy jak usnadnit uživateli zadávání a práci s větším objemem dat. Platforma je koncipována, tak aby bylo možné jednoduše a rychle rozšířit jakoukoliv část.
\section*{Summary}
This diplma thesis deals with creating platform which serves for easy manipulation with large data set. There are numerous technical things described in this thesis to understand web development. Later there are proposed approaches of how to make as easy as possible for user to define and work with large data sets. Platform is written and created in a way, that it is easy to extend eny part of it.

\hbox{}
\vfill

%------------------------------------------------------------------------------------------------------------------------------------------
%  Klíčová slova
%------------------------------------------------------------------------------------------------------------------------------------------
\section*{Klíčová slova}
big data, podpůrný nástroj, webová aplikace, uživatelské rozhraní
\section*{Key words}
big data, supportive tool, web application, user interface
\vskip1cm

\newpage
\hbox{}
\vfill

%------------------------------------------------------------------------------------------------------------------------------------------
%  Bibliografická citace
%------------------------------------------------------------------------------------------------------------------------------------------
\section*{Bibliografická citace}
HALA, K. \textit{Platforma pro definici a zpracování dat}. Brno: Vysoké učení technické v Brně, Fakulta podnikatelská, 2017. 83 s. Vedoucí diplomové práce Ing. Jiří Kříž, Ph.D..
\vskip1cm

\newpage
\hbox{}
\vfill

%------------------------------------------------------------------------------------------------------------------------------------------
%  Prohlášení
%------------------------------------------------------------------------------------------------------------------------------------------
\section*{Čestné prohlášení}
Prohlašuji, že předložená diplomová práce je původní a zpracoval jsem ji samostatně. Prohlašuji, že citace použitých pramenů jsou úplné a že jsem ve své práci neporušil autorská práva (ve smyslu Zákona č. 121/2000 Sb., o~právu autorském a~o~právech souvisejících s~právem autorským).
\\
\par V~Brně dne xx. x. 2017 % doplnit datum
\hfill\dotuline{\blank{5cm}}\hskip2cm
\par\hfill jméno \hskip5cm\blank{-4cm}
\vskip1cm

\newpage
\hbox{}
\vfill

%------------------------------------------------------------------------------------------------------------------------------------------
%  Poděkování
%------------------------------------------------------------------------------------------------------------------------------------------
\section*{Poděkování}
Rád bych poděkoval panu Ing. Jiřímu Křížovi, Ph.D. MBA za vedení a konzultace během této práce. Dále bych chtěl poděkovat rodině a partnerce za morální podporu.
\vskip1cm

%------------------------------------------------------------------------------------------------------------------------------------------
%  Obsah
%-----------------------------------------------------------------------------------------------------------------------------------------
\renewcommand{\baselinestretch}{1}
\tableofcontents

\addtocontents{toc}{\protect\thispagestyle{empty}}

%------------------------------------------------------------------------------------------------------------------------------------------
%  Vlastní texty
%------------------------------------------------------------------------------------------------------------------------------------------
\renewcommand{\baselinestretch}{1.5}

%------------------------------------------------------------------------------------------------------------------------------------------
% Úvod
%------------------------------------------------------------------------------------------------------------------------------------------
\chapter*{ÚVOD}
\addcontentsline{toc}{chapter}{ÚVOD} 
\par V aktuální době je velké množství dokumentů ve firmách zpracováváno nástroji, které do jisté míry nepodporují spolupráci a často dochází k vytváření velkých a nepřehledných dokumentů, které se časem zvětšují a po několika letech se musí kompletně přepsat a případně zrušit.

\par S nástupem internetu tento problém částečně vymizel díky tomu, že se nyní dají takové dokumenty velice snadno sdílet, nicméně stále je zde problém, že velká část nástrojů pracujících s dokumenty nenabízí jednoduché a přehledné provázání.

\par Proto se v rámci této práce zaměříme na možná řešení některých problémů, které vznikají při sdílení dokumentů a při jejich propojování. Cílem této práce je tedy analýza současných nástrojů a následně návrh a vytvoření specifického nástroje, který se bude zaměřovat z velké části na jeho snadné používání uživateli.

\par Práce nás postupně provede několika kapitolami, kdy se nejdříve budeme věnovat teoretickým východiskům, kde si vysvětlíme některé pojmy zabývající se práci s dokumenty pomocí webových aplikací, poté se zaměříme na aktuální stav nástrojů a aplikací zabývající se touto problematikou. Část popisu aktuálního stavu se budeme věnovat nástrojům, které nám mohou pomoci při samotném vývoji aplikace. Na samotný návrh a vývoj aplikace se zaměříme v poslední kapitole, ve které si rozebereme také případná rizika vývoje takového nástroje.

%------------------------------------------------------------------------------------------------------------------------------------------
% Cíl a metodika práce
%------------------------------------------------------------------------------------------------------------------------------------------
\chapter*{CÍL DIPLOMOVÉ PRÁCE}
Hlavním cílem této práce je vytvořit platformu, která bude schopná zpracovat data zadaná uživatelem, analyzovat je a na základě vnitřní logiky a informace nesené v těchto datech je uložit do logických celků. Tyto celky poté umožní zobrazit uživateli, nechá jej dále definovat dodatečné informace, provádět reporting a případně zobrazit v přehledných grafech.

Platforma se bude zaměřovat převážně na uživatelské rozhraní, tak aby její používání bylo co nejintuitivnější a nejjednodušší.
\addcontentsline{toc}{chapter}{CÍL DIPLOMOVÉ PRÁCE}

\chapter*{METODIKA PRÁCE}
\addcontentsline{toc}{chapter}{METODIKA PRÁCE}
\section*{Metody}
\par Nejdůležitějším zaměřením této platformy je uživatelská přívětivost a jednoduchost na používání, proto bude při vývoji kladen důraz na spokojenost uživatelů. Tohoto bude dosaženo použitím agilních metodik při vývoji, kdy bude postupně dodávaný produkt předáván úzkému kruhu uživatelů, kteří se budou vyjadřovat k uživatelskému rozhraní. Platforma bude psána jako webová aplikace, která bude přistupovat do databáze přes rozhraní napsané v jazyce Java.
\par V části zpracování dat bude použito několik ETL metodik a data mining technik, které povedou k získání logických informací ze zadaných dat. Platforma bude vyvíjena s možností 
škálovatelnosti a použití nad velkým objemem dat.
\addcontentsline{toc}{section}{Metody}

\section*{Postupy}
\addcontentsline{toc}{section}{Postupy}
\par K vytvoření co nejpřívětivější platformy budou využity naše zkušenosti a knižní publikace zabývající se tímto tématem. Dále budou analyzovány jednotlivé postupy zadávání dat uživatelů do tohoto systému, které povedou ke zpřehlednění a zjednodušení používání.
\par Pro komunikaci se serverem bude použit standard REST, který usnadní komunikaci se serverem a umožní případné navázání nových aplikací. V případě, že bude vytvořena mobilní aplikace pro získávání dat, nebude nutné psát znovu stejnou nebo podobnou logiku.
\par Pro zabezpečené přihlášení do aplikace bude použit autentizační server, který bude zajišťovat vytváření a správu uživatelů spolu s jejich právy.



\pagestyle{plain}     % zapne obyčejné číslování

%------------------------------------------------------------------------------------------------------------------------------------------
%  Teorie
%------------------------------------------------------------------------------------------------------------------------------------------
\chapter{TEORETICKÁ VÝCHODISKA}
\par Pro plné pochopení, výběru a případném vypracování platformz je potřeba si objasnit a vysvětlit několik témat. Jsou to především \textit{Vývojové platformy low-code}, výsledná platforma by měla splňovat tuto definici. Dále si objasníme pojemy \textit{Bussiness intelligence} (platforma bude z části pracovat s touto oblastí) a \textit{Platforma pro pokročilou vizualizaci dat} -- pro snadné používání uživatelského rozhraní. A vzhledem k tomu že výsledná platforma musí do určité části pracovat s uživatelskými právy a spravovat uživatele, objasníme si pojem \textit{Server pro řízení přístupu a identity}

\section{Vývojové platformy low-code}
\par Vývojové platformy low-code jsou celkem nový pojem, tyto produkty začali vznikat, protože malé a střední podniky potřebovali vytvořit rychle a za použití menšího počtu vývojářů aplikace, které mohou být nadále rychle spravovány. \cite{pcmag-no-coding}

\par Toto v podstatě znamená, že vývojáři mohou rychle měnit software na základě uživatelských požadavků, což má za následek spokojenější uživatele, uživatelsky přívětivější software a toto všechno za minimálního použití ručního programování. Takovéto platformy neeliminují programování jako takové, ale napomáhají rychlejšímu vývoji, tak že poskytují vizuální nástroje a napomáhají konfiguraci datových modulů a pomáhají eliminovat problémy spojené s  datovou integrací. \cite{low-code-customer-want}

\paragraph{Výhody low-code platforem}
\begin{itemize}
  \item \textbf{Produktivita:} Systémy mohou být vyvýjeny a nasazeny během menšího časového rozmezí, oproti klasickému programování. \cite{low-code-accelerate}
  \item \textbf{Reakční schopnost:} Vývojář může často zvolit různé druhy platforem na kterých bude výsledný produkt fungvat, od mobilních aplikací, až po webové služby. \cite{low-code-accelerate}
  \item \textbf{Spolehlivost:} Aplikace mohou být aktualizovány mnohem rychleji, což má za následek jejich stabilitu a spolehlivost. \cite{low-code-accelerate}
  \item \textbf{Úspora času a peněz:} Vývojáři mohou vytvořit mnohem více funkcionality za kratší čas, z čehož plyne že si firma může dovolit mensí počet programátorů. \cite{low-code-accelerate}
  \item \textbf{Zaměření na samotný vývoj:} Zaměřením na to co má aplikace dělat, a ne jak to má dělat, programátoři se mohou zaměřit na funkcionalitu a uživatelskou spokojenost. Při vývoji je možné se zaměřit také více na uživatelské požadavky mnohem rychleji. \cite{low-code-accelerate}
\end{itemize}

\subsection{Příklady low-code platforem}
\paragraph{Microsoft PowerApps} Vývojová platforma od firmy Microsoft, která dovoluje vytvořit během několika málo kliknutí aplikaci pro mobilní platformy a také jako webové služby. Při spojením této platformy a aplikace Power BI vzniká velice robustní vývojářský nástroj, díky kterému je možné rychle integrovat produkční data do aplikace, kterou budou uživatelé rádi používat. \cite{pcmag-no-coding}
\paragraph{Zoho Creator} Výhodou této platformy je využití techniky \uv{\tt{drag-and-drop}}, která umožňuje vytvářet aplikace a převážně jejich uživatelské rozhraní bez nutnosti psát jakýkoliv kód. \cite{zoho-review}
\paragraph{Rollbase} Při používání této platformy vývojář jako první definuje objekty, jejich vlastnosti a vztahy mezi těmito objekty. Po překonání tohoto kroku máme již plně funkční webovou aplikaci, která je funkční napříč všemi mobilními zařízeními. \cite{what-is-low-code}
\paragraph{Openshift} Platforma pro vývoj webovým a mobilních aplikací, postavená na kontejnerech, které zajišťuí rychlý vývoj a možnost dedikovat vývojáře na vytvoření jednoduchých funcionalit jako samostatné aplikace \footnote{Takovýmto aplikacím se říka Microservice \url{https://smartbear.com/learn/api-design/what-are-microservices/}}, které za pomocí Openshiftu vytvoří velkou a komplexní aplikaci.\cite{openshift-overview}
\begin{figure}[h]
\centering
\includegraphics[width=\textwidth]{openshift}
\caption{Znozornění jednotlivých vrstev v platformě openshift.}
\end{figure}

\subsection{Platforma jako služba -- Paas}

\section{Bussiness Intelligence}

\subsection{Big data}

\subsection{Datamining}

\subsection{Extraction, Transaction, Loading}

\subsection{NoSql databáze}
\cite{nosql}

\section{Platforma pro pokročilou vizualizaci dat}
\subsection{Single page aplikace}
\subsection{Dynamická a interaktivní vizualizace dat}
\subsection{Webová služba RESTful}
Restufl web APIs

\section{Server pro řízení přístupu a identity}
\url{http://www.sersc.org/journals/IJMUE/vol9_no9_2014/9.pdf}

\subsection{JSON Web Token}
https://tools.ietf.org/html/rfc7519
https://scotch.io/tutorials/the-anatomy-of-a-json-web-token

%------------------------------------------------------------------------------------------------------------------------------------------
%  Analyza prostredku
%------------------------------------------------------------------------------------------------------------------------------------------
\chapter{ANALYTICKÁ ČÁST} \label{analyza}
\par 
%------------------------------------------------------------------------------------------------------------------------------------------
%  Navrhy
%------------------------------------------------------------------------------------------------------------------------------------------
\chapter{VLASTNÍ NÁVRHY}
\par Nástroj, který je součástí této práce se zaměří z velké části na uživatelské rozhraní a do jisté části využije technologie a nástroje, které jsou dostupné a usnadní takto vývoj a nasazení tohoto nástroje. Dále se zaměříme na životaschopnost tohoto nástroje a způsobu jakým bude takto vytvořený nástroj nabízen veřejnosti. Rozsah firem, pro které bude tato aplikace doporučována je malé, až střední podniky, které pracují s větším množstvím dat a potřebují nějakým způsobem najít zkryté informace.

\section{Návrh aplikace}
\par Aplikace je postavena na základě aplikačního serveru WildFly, který pracuje s moduly napsanými v jazyce Java. Základní rozdělení aplikace je tedy \textbf{Engine} a \textbf{UI}, engine je připojen na databázi MongoDb, což je NoSql databáze. Výhodou tohtoto nastavení je vysoká pružnost v rámci zapsaných dat a není tedy nutné  přesně definovat závislosti v rámci databáze (každý zákazník si může definovat jiné důležité atributy pro každý projekt).
\par Další výhodou oddělení uživatelského rozhraní a samostaného enginu je možnost lépeřídit vývojářský tým, lépe rozdělit práci a v neposledn řadě, také možnost později vytvořit klienta nezávislého na dosavadním uživatelském rozhraní -- například vytvoření mobilní aplikace, která se bude specializovat na určitou část systému. Celý sysém spolu komunikuje tedy pomocí REST rozhraní a o zabezpečení se stará vrstva Keycloak, ve kterém jsou oba moduly registrovány a slouží jako takový středobod celého systému. Jak jsou jednotlivé části propojeny můžete vidět na obázku \ref{schema}, kde hrany znamenají komunikační zprávy a uzly logické bloky.

\begin{figure}[htp]
  \centering
  \includegraphics[max size={\textwidth}]{placeholder.pdf}
  \caption{Diagram znázorňující schéma nástroje.}
  \label{schema}
\end{figure}

\par Pokud uživatel již disponuje velkým datovým uložištěm je ho možné připojit do systému jednoduchou konfigurací v rámci UI, systém prozkoumá tyto data, některá si nakopíruje a začne je uživateli nabízet podobně, jako kdyby je uživatel měl již dříve v systému. Podobně je tomu s dolováním dat -- modul, který je zodpovědný za samostatné dolování je možné připojit do systému, odkázat jej na datové uložiště a spustit samostatné dolování dat. Vše je nastavitelné jak z uživatelského prostředí, tak pomocí volání na předem definované RESTové služby.

\section{Cenový model}

\chapter*{ZÁVĚR}
\par Cílem této práce bylo vytvořit platformu, která bude schopná pracovat s daty, vyhodnocovat jejich vnitřní informaci a následně nabízet uživateli jednoduché rozhraní pro práci s těmito daty.

\par Nejdříve bylo potřeba se seznámit s několika teoretickými pojmy, které byli důležité pro pochopení celého problému vývoje a návrhu takto rozsáhlé platformy. Vzhledem k tomu, že takováto platforma bude muset být často nasazena u několika zákazníků zároveň a že práce musí být plynulá a nemělo by záležet kde a jak ji uživatelé používají, proto bylo zvoleno použití webových technologií a k nim přidruženým nástrojům.

\par V rámci návrhu a vývoje platformy došlo k několika testování u různě schopných uživatelů, což mělo za následek částečnou změnu uživatelského rozhraní, a do podoby, která je snadná na používání a jednoduše pochopitelná.

\par Hlavní výhodou tohoto nástroje je jeho snaha o pochopení vnitřní informace v datech, které uživatel poskytne. Nad těmito daty se následně provádí několik funkcí, které nám pomohou s následným zpracováním tak, že uživateli jsou automaticky napovídány hodnoty, které již v systému jsou zapsány. Tímto se značně usnadňuje uživateli práce jak při zadávání nových hodnot, tak práce s již existujícími datovými sadami.

\par Během psaní této práce byl již nástroj představen několika případným investorům, kteří vyslovili velký zájem a chtěli by do něj přenést svá data. Někteří investoři sami chválí myšlenku této platformy a jsou spokojeni s jednoduchostí a snadným používáním.
\addcontentsline{toc}{chapter}{ZÁVĚR}
\newpage

\listoffigures
\addcontentsline{toc}{chapter}{Seznam obrázků}
\newpage
\listoftables
\addcontentsline{toc}{chapter}{Seznam tabulek}
\newpage
\addcontentsline{toc}{chapter}{Seznam rovnic}
\newpage
\listofequationcaps
\addcontentsline{toc}{chapter}{Seznam grafů}
\newpage
\listofgraphs
\addcontentsline{toc}{chapter}{Seznam zkratek}
\glsaddall
\printglossary[title=Seznam zkratek, nonumberlist]
\addcontentsline{toc}{chapter}{Seznam použitých zdrojů}
\begin{thebibliography}{99}

\bibitem{1}{citace}\label{odkaz}



\end{thebibliography}



\end{document}
