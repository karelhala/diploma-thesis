%------------------------------------------------------------------------------------------------------------------------------------------
%  Navrhy
%------------------------------------------------------------------------------------------------------------------------------------------
\chapter{VLASTNÍ NÁVRHY}
\par Nástroj, který je součástí této práce se zaměří z velké části na uživatelské rozhraní a do jisté části využije technologie a nástroje, které jsou dostupné a usnadní takto vývoj a nasazení tohoto nástroje. Dále se zaměříme na životaschopnost tohoto nástroje a způsobu jakým bude takto vytvořený nástroj nabízen veřejnosti. Rozsah firem, pro které bude tato aplikace doporučována je malé, až střední podniky, které pracují s větším množstvím dat a potřebují nějakým způsobem najít zkryté informace.

\section{Návrh aplikace}
\par Aplikace je postavena na základě aplikačního serveru WildFly, který pracuje s moduly napsanými v jazyce Java. Základní rozdělení aplikace je tedy \textbf{Engine} a \textbf{UI}, engine je připojen na databázi MongoDb, což je NoSql databáze. Výhodou tohtoto nastavení je vysoká pružnost v rámci zapsaných dat a není tedy nutné  přesně definovat závislosti v rámci databáze (každý zákazník si může definovat jiné důležité atributy pro každý projekt).
\par Další výhodou oddělení uživatelského rozhraní a samostaného enginu je možnost lépeřídit vývojářský tým, lépe rozdělit práci a v neposledn řadě, také možnost později vytvořit klienta nezávislého na dosavadním uživatelském rozhraní -- například vytvoření mobilní aplikace, která se bude specializovat na určitou část systému. Celý sysém spolu komunikuje tedy pomocí REST rozhraní a o zabezpečení se stará vrstva Keycloak, ve kterém jsou oba moduly registrovány a slouží jako takový středobod celého systému. Jak jsou jednotlivé části propojeny můžete vidět na obázku \ref{schema}, kde hrany znamenají komunikační zprávy a uzly logické bloky.

\begin{figure}[htp]
  \centering
  \includegraphics[max size={\textwidth}]{placeholder.pdf}
  \caption{Diagram znázorňující schéma nástroje.}
  \label{schema}
\end{figure}

\par Pokud uživatel již disponuje velkým datovým uložištěm je ho možné připojit do systému jednoduchou konfigurací v rámci UI, systém prozkoumá tyto data, některá si nakopíruje a začne je uživateli nabízet podobně, jako kdyby je uživatel měl již dříve v systému. Podobně je tomu s dolováním dat -- modul, který je zodpovědný za samostatné dolování je možné připojit do systému, odkázat jej na datové uložiště a spustit samostatné dolování dat. Vše je nastavitelné jak z uživatelského prostředí, tak pomocí volání na předem definované RESTové služby.

\section{Cenový model}
