\chapter*{ZÁVĚR}
\par Cílem této práce bylo vytvořit platformu, která bude schopná pracovat s daty, vyhodnocovat jejich vnitřní informaci a následně nabízet uživateli jednoduché rozhraní pro práci s těmito daty.

\par Nejdříve bylo potřeba se seznámit s několika teoretickými pojmy, které byli důležité pro pochopení celého problému vývoje a návrhu takto rozsáhlé platformy. Vzhledem k tomu, že takováto platforma bude muset být často nasazena u několika zákazníků zároveň a že práce musí být plynulá a nemělo by záležet kde a jak ji uživatelé používají, proto bylo zvoleno použití webových technologií a k nim přidruženým nástrojům.

\par V rámci návrhu a vývoje platformy došlo k několika testování u různě schopných uživatelů, což mělo za následek částečnou změnu uživatelského rozhraní, a do podoby, která je snadná na používání a jednoduše pochopitelná.

\par Hlavní výhodou tohoto nástroje je jeho snaha o pochopení vnitřní informace v datech, které uživatel poskytne. Nad těmito daty se následně provádí několik funkcí, které nám pomohou s následným zpracováním tak, že uživateli jsou automaticky napovídány hodnoty, které již v systému jsou zapsány. Tímto se značně usnadňuje uživateli práce jak při zadávání nových hodnot, tak práce s již existujícími datovými sadami.

\par Během psaní této práce byl již nástroj představen několika případným investorům, kteří vyslovili velký zájem a chtěli by do něj přenést svá data. Někteří investoři sami chválí myšlenku této platformy a jsou spokojeni s jednoduchostí a snadným používáním.