\chapter*{ZÁVĚR}
\par Cílem této práce bylo vytvořit platformu, která bude schopná pracovat s daty, vyhodnocovat jejich vnitřní informace a následně nabízet uživateli jednoduché rozhraní pro práci s těmito daty.

\par Nejdříve bylo potřeba seznámit se s několika teoretickými pojmy, které byly důležité pro pochopení celého problému vývoje a návrhu rozsáhlé platformy. Vzhledem k tomu, že taková platforma bude muset být nasazena u několika zákazníků zároveň a že práce musí být plynulá, nemělo by záležet kde a jak ji uživatelé používají. Proto bylo zvoleno použití webových technologií a k nim přidružených nástrojů, které umožňují snažší přístup pro větší množství uživatelů.

\par V rámci návrhu a vývoje platformy došlo několikrát k testování u různě schopných uživatelů, což mělo za následek částečnou změnu uživatelského rozhraní do podoby, která je  jednoduše pochopitelná a snadná na používání.

\par Hlavní výhodou tohoto nástroje je jeho snaha o pochopení vnitřních informací v datech, které uživatel poskytne. Nad těmito daty se následně provádí několik funkcí, které nám pomohou s následným zpracováním tak, že uživateli jsou automaticky napovídány hodnoty, které jsou již v systému zapsány. Tímto se značně usnadňuje uživateli práce jak při zadávání nových hodnot, tak práce s již existujícími datovými sadami.

\par Během psaní této práce byl nástroj již představen několika případným investorům, kteří o něj vyslovili zájem a chtěli by do něj přenést svá data. Někteří investoři sami chválí myšlenku nástroje, který kombinuje webové technologie a práci s tabulkami. Nejvíce si však pochvalují jednoduchost a snadné používání vytvářené aplikace.